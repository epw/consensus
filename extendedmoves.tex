% -*- Mode: latex -*-
\documentclass[letterpaper,10pt]{article}

\usepackage{multicol}
\usepackage{geometry}
\usepackage{xltxtra}
\usepackage{color}
\usepackage{fontspec}
\usepackage[compact]{titlesec}
\usepackage{setspace}
\usepackage{tikz}
\usepackage{graphicx}
\usepackage{textpos}
\usepackage{lipsum}
\usepackage{fancyhdr}
\usepackage{comment}

\geometry{letterpaper, landscape}

\setlength{\columnseprule}{1pt}

\setmainfont [Path = fonts/,
  UprightFont = Aller_Rg.ttf,
  BoldFont = Aller_Bd.ttf,
  ItalicFont = Aller_It.ttf
]{Aller}
\newfontfamily\titlefont [Path = fonts/,
  UprightFont = Kingthings_Printingkit.ttf,
]{Kingthings}

\titlespacing{\section}{0pt}{*0}{*0}
\titlespacing{\subsection}{0pt}{*0}{*0}

\setcounter{secnumdepth}{-1}

\newenvironment{move}{}{}
\excludecomment{movedetail}
\newcommand{\TRIGGER}[1]{\textbf{#1}}
\newcommand{\SEPARATOR}{\vspace{-1em}\begin{center}\noindent\rule{6cm}{2pt}\end{center}}

\pagestyle{fancyplain}
\renewcommand{\headrulewidth}{0pt}
\setlength{\columnsep}{.3in}
\setlength{\oddsidemargin}{-.9in}
\setlength{\topmargin}{-.75in}
\setlength{\textwidth}{10.7in}
\setlength{\textheight}{10in}
\cfoot{}


\chead{\titlefont\Huge\textbf{Extended Moves}}

\begin{document}
\begin{multicols}{3}
  \begin{move}
    When you fail a roll on a magical action and decide to \TRIGGER{put
      willpower behind changing it}, the MC may ask you some
    questions. Answer them honestly. Re-roll the roll with +1 and choose
    2:
    \begin{itemize}
      \setlength\itemsep{0em}
    \item Sever your connection to an Anchor
    \item Take -1 ongoing until you actually fail a magical roll
    \item You \TRIGGER{Backlash}
    \end{itemize}
  \end{move}

  \SEPARATOR
  
  \begin{move}
    When you fail a roll with no magical influence, and decide to
    \TRIGGER{use magic to fix it}, treat the result plus your stat as a
    7, and choose 2:
    \begin{itemize}
      \setlength\itemsep{0em}
    \item Take 2 harm, or 2 Discord
    \item The magic is Rending
    \item Lose a Playbook move until the end of session
    \end{itemize}
  \end{move}

  \SEPARATOR

  \begin{move}
    When you \TRIGGER{suffer harm} (even 0 harm), roll just +Harm
    suffered. On a 10+, the MC can choose 1:
    \begin{itemize}
      \setlength\itemsep{0em}
    \item You're out of action: unconscious, trapped, incoherent, or
      panicked.
    \item Take the full Harm of the attack, before it was reduced by
      preparations. If you already took the full Harm of the attack, take
      +1 Harm.
    \item You are shaken. Take -1 ongoing until you can spend time
      connecting with an Anchor.
    \end{itemize}
    On a 7-9, the MC can choose 1:
    \begin{itemize}
      \setlength\itemsep{0em}
    \item You lose your footing
    \item You lose your grip on whatever you're holding
    \item You let something or someone you're attending to fall into
      danger, or drop the ball on an obligation (especially important
      if it affects an Anchor)
    \item You are delayed, to deal with the effects of your injury
    \item Something you weren't prepared for happens
    \end{itemize}
    On a 6-, you take the Harm, but things don't get worse.
  \end{move}

  \columnbreak
  
  \begin{move}
    When you have some time and relative safety and you \TRIGGER{plan
      a ritual} of magical power, describe the effect you are trying
    for. The MC may say more information or clariy is needed before
    the plan can be finalized. Once those needs are met, the MC will
    say which of these are needed to perform the ritual (it may be
    more than one):
    \begin{itemize}
      \setlength\itemsep{0em}
    \item Extra time
    \item A certain object
    \item Help from an outside source
    \end{itemize}
    Then, write down the plan and hold 1.
  \end{move}

  \SEPARATOR

  \begin{move}
    When you have time, relatively safety, a place of power, and
    enough Mages, and you \TRIGGER{begin a ritual}, have each Mage
    involved describe how they will contribute to the ritual within
    their Paradigm. if you spend a hold from Plan a Ritual, tell the
    MC the effect that will occur on a success, otherwise, the MC
    decides based on each Mage's contributions. Take into onsideration
    each participant's Opposed, and the probability of Rending
    magic. Then, roll +Will. On a 10+, the ritual works as
    expected. On a 7-9 the MC will choose at least one.
    \begin{itemize}
      \setlength\itemsep{0em}
    \item The ritual takes longer than expected
    \item You draw unwanted attention
    \item The ritual causes Discord
    \item Everyone involved Backlashes
    \item The ritual has greater than intended effects
    \end{itemize}
  \end{move}

  \SEPARATOR

  \begin{move}
    When you \TRIGGER{lose an Anchor}, roll +Anchors left. On a 10+,
    it may take some time, but you'll get through this. On a 7-9,
    choose 1:
    \begin{itemize}
      \setlength\itemsep{0em}
    \item Another Anchor gets put in danger
    \item You do something you will regret. The MC tells you
      what. \textit{Note: This means it does \textbf{not} cause
        another Anchor to get put in danger}
      \item You Backlash
    \end{itemize}
    On a miss: Lose another Anchor. The MC will tell you which one and
    how.
  \end{move}

  \columnbreak

  \begin{move}
    At the \TRIGGER{end of session}: note any effects that lasted
    ``until end of session,'' but should continue into the next. Then,
    ask yourselves these questions as a group. For every one you
    answer ``yes'' to, everyone marks experience.
    \begin{itemize}
      \setlength\itemsep{0em}
    \item Did you show the lengths you were willing to go to in order
      to protect an Anchor?
    \item Did you uncover a secret?
    \item Did you learn something that puts your paradigms in
      perspective, or caused you to question them?
    \item Did you expose the depths of your humanity, to yourself or
      to someone else?
    \item Did you witness loss, selfishness, or pain born of magic?
    \end{itemize}
  \end{move}

  \SEPARATOR
  
  \begin{move}
    When your \TRIGGER{Harm reaches Code N}, roll +Body. On a 10+ you
    stabilize, and will need medical care, but you should live. On a
    7-9, some serious magic is going to be needed to help you survive,
    plus the medical care. On a 6-, you're about to bite the
    dust. Make peace with your Paradigm.

    If you have two or less Anchors, you have the option of removing
    an Anchor in order to stabilize. Describe how you spiritually cut
    yourself off from the Anchor, infusing yourself with magic in
    order to heal, but taking a huge step away from reality in doing
    so.
  \end{move}

  \SEPARATOR

  \begin{move}
    When \TRIGGER{one of your Anchors is put in danger}, mark
    experience and either go and handle it, or lose them as an
    Anchor.
  \end{move}

  \SEPARATOR

  \begin{move}
    When you \TRIGGER{describe a magical effect within your Paradigm
      but beyond your normal abilities}, the MC may tell you a
    resource you could consume to achieve the effect.
  \end{move}

\end{multicols}
\end{document}
